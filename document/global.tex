\usepackage[utf8]{inputenc} %UTF8 nutzen
\usepackage[ngerman]{babel} %deutsch umlaute

\usepackage[a4paper,left=2.5cm,right=2.5cm, top=2cm, bottom=2.5cm]{geometry}
%Zeilenabstand anpassen
\usepackage{setspace}
%\onehalfspacing
%Neues Seiten format, bitte
\setlength{\footskip}{1.2cm}

\usepackage{fancyhdr} %Huebsche footer und header
\usepackage[bookmarks=true]{hyperref} %fuer klickbare links und url 
%formatierungen
\hypersetup{pdfborder={0 0 0}} %--rahmen um links in pdf ausschalten
%\usepackage{showframe}

\usepackage{bibgerm}
%\usepackage[superscript]{cite} %Hochgestellte Zahlen für quellen

\usepackage{multirow}
\usepackage{rotfloat}
\usepackage{tabularx}
\usepackage{selinput}
\usepackage{listings}
\usepackage{lmodern}
\usepackage{paralist}
\usepackage{array}
\usepackage{float}
\usepackage[T1]{fontenc}
\usepackage{diagbox}
\usepackage[]{pgfgantt}
\usepackage[]{palatino}
\usepackage{csquotes}
\usepackage{paralist}
% hyphenations
%\hyphenation{}



\renewcommand{\arraystretch}{1.5}
\newcommand{\versiondate}{07.01.2015}
\newcommand{\authorname}{Danny Koppenhagen}

%Verzeichnisse
\usepackage[resetlabels]{multibib}
\newcites{que}{Quellenverzeichnis}
\newcites{lit}{Literaturverzeichnis}

\usepackage{listings} %für quellcode

\usepackage{graphicx} %grafiken einbinden
%\usepackage{warpfig} % Text um Bild

\usepackage{amssymb}
%Beispiele fuer manuelle Silbentrennung
% Silbentrennung direkt mit /- oder /"
%\hyphenation{ge-wuensch-ten}

\usepackage[printonlyused]{acronym} %für Abkürzungsverzeichnis und 
%Abkürzungsverwendung

% include pdfs directly into tex
\usepackage{pdfpages}
\usepackage[section]{placeins}
% LISTINGS
\usepackage{color} % Wird benoetigt fuer listing
\usepackage{xcolor} % Wird benoetigt fuer listing

\definecolor{dkgreen}{rgb}{0,0.6,0}
\definecolor{gray}{rgb}{0.5,0.5,0.5}
\definecolor{lightgray}{gray}{0.5}
\definecolor{mauve}{rgb}{0.58,0,0.82}
\definecolor{darkgray}{rgb}{0.4,0.4,0.4}
\definecolor{purple}{rgb}{0.65, 0.12, 0.82}
\definecolor{orange}{rgb}{1, 0.3, 0}
